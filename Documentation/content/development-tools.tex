\section{Development Tools}
\label{sec:development-tools}

This section describes all used third party tools and libraries.
First all common tools, then the frontend libraries and finally all backend libraries are listed.
Currently, the project uses the \href{https://mad-srv.informatik.uni-erlangen.de/}{GitLab Server} of the \href{https://www.mad.tf.fau.de/}{MaD Lab}.

\subsection{Common tools}

\paragraph{Git repository}
Currently, the project uses the \href{https://mad-srv.informatik.uni-erlangen.de/}{GitLab Server} of the \href{https://www.mad.tf.fau.de/}{MaD Lab}.

\paragraph{NodeJS}
\href{https://nodejs.org}{NodeJS} is a JavaScript runtime that must be installed to develop, test and build the project.
It is also required for running the backend.

\paragraph{Linter}
\href{https://eslint.org}{ESLint} statically analyzes our code to find common problems (for example accessing an undefined variable). 
ESLint is configured with the \texttt{.eslintrc.js} file. 
More rules can be added to ESLint via plugins. 
The following plugins are used:
\begin{itemize}
    \item \href{https://www.npmjs.com/package/eslint-plugin-import}{eslint-plugin-import}: 
      This plugin intends to support linting of ES2015+ (ES6+) import/export syntax and prevent issues with misspelling of file paths and import names. 
      It also enforces the ordering of import statements.
    \item \href{https://www.npmjs.com/package/eslint-plugin-prettier}{eslint-plugin-prettier}:
      This plugin runs our \nameref{sec:code-formatter} prettier inside of ESLint. 
      It also uses the \href{https://www.npmjs.com/package/eslint-config-prettier}{eslint-config-prettier} configuration to disable rules of ESLint which conflict with prettier.
    \item \href{https://www.npmjs.com/package/eslint-plugin-jsx-a11y}{eslint-plugin-jsx-a11y}: This plugin adds rules for \href{https://reactjs.org/docs/introducing-jsx.html}{React's JSX} syntax.
    \item \href{https://www.npmjs.com/package/eslint-plugin-react}{eslint-plugin-react}: This plugin adds react specific rules.
    \item \href{https://www.npmjs.com/package/eslint-plugin-react-hooks}{eslint-plugin-react-hooks}: This plugin enforces React's \href{https://reactjs.org/docs/hooks-rules.html}{Rules of Hooks}.
\end{itemize}

\paragraph{Code Formatter}
\label{sec:code-formatter}
\href{https://prettier.io}{Prettier} is used to have a consistent code formatting in your code. 
To format the code run the \texttt{npm run lint-fix} command.

\paragraph{Continous Integration (CI)}
We are using \href{https://mad-srv.informatik.uni-erlangen.de/help/ci/README.md}{GitLab CI} to run the linter, the tests and the build on every push onto the GitLab Server. 
The jobs are configured inside the \texttt{.gitlab-ci.yml} file. 
Every job must have the tag \texttt{docker} to be executed.

\subsection{Frontend libraries}

\paragraph{Frontend Javascript framework}
The frontend is written in Javascript and uses the \href{https://reactjs.org}{React} framework.
The project was initially created with the \href{https://create-react-app.dev}{Create React App} tool.

\paragraph{CSS framework}
\href{https://getbootstrap.com}{Bootstrap} is used as a CSS library for nice UI Elements. 
To use the bootstrap components like normal React components the library \href{https://react-bootstrap.github.io}{react-bootstrap} is used.

\paragraph{Router}
\href{https://reacttraining.com/react-router/web/guides/quick-start}{React-router-dom} is used to provide support for handling different routes in our frontend (e.g. \texttt{/} and \texttt{/editor}).

\paragraph{Icons}
\href{https://fontawesome.com}{Fontawesome} provides lots of free icons.

\paragraph{HTTP client}
We use \href{https://github.com/axios/axios}{Axios} as our HTTP client.
It is used to access our REST API of the backend, to save and load stress test configurations.

\paragraph{Drag and drop}
\href{https://github.com/atlassian/react-beautiful-dnd}{react-beautiful-dnd} provides beautiful and accessible drag and drop for lists.
It is used extensively in our editor.

\paragraph{Translation}
\href{https://www.i18next.com}{I18next} is an internationalization-framework written in and for JavaScript, which we use to provide translations. 
The \href{https://react.i18next.com}{react-i18next} library adds React specific components like React Hooks for i18next. 
Currently, only English is available as language. 
The translations are specified inside \texttt{public/locales/en/stress-app.json} and are loaded via HTTP with the \href{https://github.com/i18next/i18next-http-backend}{i18next-http-backend} library.

The following example shows the usage of \texttt{useTranslation()} hook from react-i18next. 
This hook returns a \texttt{t} function, which takes a translation key as an argument and returns the translation.
\begin{verbatim}
import {useTranslation} from "react-i18next";
function Component() {
  const {t} = useTranslation();
  return <span>{t("title")}</span>;
}
\end{verbatim}

\paragraph{Copy to clipboard}
The library \href{https://github.com/nkbt/react-copy-to-clipboard}{react-copy-to-clipboard} is used to easily copy content to the clipboard.

\paragraph{Generating ids}
The library \href{https://github.com/uuidjs/uuid}{uuid} is used to create \href{https://www.ietf.org/rfc/rfc4122.txt}{RFC4122} UUIDs. 
These unique ids are used for identifying each screen and overlay in a stress pipeline.

\paragraph{Tests}
For frontend tests, we use the tools that are preconfigured by create-react-app. 
It uses \href{https://jestjs.io/}{jest} as a javascript testing framework and the \href{https://testing-library.com}{testing library} as a testing utility. 
\href{https://www.npmjs.com/package/@testing-library/react}{@testing-library/react} adds support for testing react components, 
\href{https://www.npmjs.com/package/@testing-library/jest-dom}{@testing-library/jest-dom} adds custom DOM element matchers for jest and 
\href{https://www.npmjs.com/package/@testing-library/user-event}{@testing-library/user-event} adds advanced browser interactions such as clicking and typing.

Tests should be placed in the same folder as the file to test and the file extension for tests must be \texttt{.test.js}.

\paragraph{precompress}
The \href{https://www.npmjs.com/package/precompress}{precompress} tool generates gzip and brotli compressed file for static web servers.

\subsection{Backend libraries}

\paragraph{Web framework}
The backend uses \href{https://expressjs.com}{Express} as the web framework for the REST API.
The REST API uses JSON to exchange data.
The \href{https://www.npmjs.com/package/body-parser}{body-parser} library is used parse the JSON body of HTTP requests.
The \href{https://www.npmjs.com/package/compression}{compression} library is used to add gzip compression to responses to save bandwith.
To be able to serve the pre-compressed static files of the frontend the \href{https://www.npmjs.com/package/express-static-gzip}{express-static-gzip} library is used.

\paragraph{Database}
To access the database the official \href{https://www.npmjs.com/package/mongodb}{mongodb} driver for NodeJS is used.
Also the \href{https://www.npmjs.com/package/mongodb-client-encryption}{mongodb-client-encryption} library must be added.

\paragraph{Bundler}
The JavaScript module bundler \href{https://rollupjs.org}{rollup} is used to create a single file, that can be executed.
In development the \href{https://www.npmjs.com/package/@rollup/plugin-run}{@rollup/plugin-run} is used to start a development server.
In production the \href{https://www.npmjs.com/package/rollup-plugin-terser}{rollup-plugin-terser} is used to execute the JavaScript mangler and compressor \href{https://terser.org}{terser}.
