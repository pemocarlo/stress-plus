\section{Deployment}
\label{sec:deployment}

The easiest way to deploy Stress+ is using a Docker container.
To build the docker image a \texttt{Dockerfile} is present in the root directory. 
The \texttt{Dockerfile} will build the frontend and configure the backend to serve the frontend. 
Run the following command in the root directory to build the docker image with the name \texttt{stress-plus}.
\begin{verbatim}
    docker build -t stress-plus .
\end{verbatim}
After the docker image is built it can be started with:
\begin{verbatim}
    docker run 
      --detach 
      --name stress-plus 
      --publish 80:80 
      --env MONGODB_URI=<mongodb-uri> 
      stress-plus
\end{verbatim}
Stress+ is now running on port $80$ of the host machine.
By default, the backend will listen on port $80$ of the container, but this can be changed by setting the \texttt{PORT} environment variable.
Because the backend requires a running MongoDB database, you must set the \texttt{MONGODB\_URI} environment variable. 
The value should have the following schema:
\begin{verbatim}
    mongodb://<username>:<password>@<host>:<port>/<database-name>
\end{verbatim}

For more information on how to run docker containers refer to the \href{https://docs.docker.com/engine/reference/commandline/run}{docker documentation}.

\subsection*{Client prerequisites}
The participants and test creators must use an up-to-date browser to access Stress+.
We recommend using Chrome or Firefox, because we have only tested with them.
Stress+ does not work with Internet Explorer.
