\section{Development}
\label{sec:development}

This section describes how to develop new screen and overlay modules.
All modules have a unique name, a React component that displays the module during the execution of the stress test and a React component to adjust the settings of the module in the editor. 
Every module must be registered in the corresponding registry and should have its own folder inside the \texttt{src/overlays} or \texttt{src/screens} directory.

\subsection{Setup development environment}
Both backend and frontend can be started on your local machine in development mode.
Before starting them, make sure that you have installed \href{https://nodejs.org}{NodeJS} and have executed \texttt{npm install} in the \texttt{Code/frontend} and the \texttt{Code/backend} directory.

In development mode the backend uses the MongoDB database referenced by the \texttt{MONGODB\_URI} variable in the \texttt{.env} file in the backend directory.
Please make sure this database is up and running or change the variable to a running MongoDB database.
Run \texttt{npm run start} in the backend directory to start the backend on port $4000$.

To start the frontend run \texttt{npm run start} in the frontend directory.
Your browser should now be opened automatically and show the start page of Stress+.
If thats not happening, navigate to \texttt{localhost:3000} in your browser to open Stress+.
You can now start developing or play around with Stress+.

\subsection{Screen Component}
\label{sec:development-screens}

Now, you will learn how to code a screen's main component. 
This React component is responsible for displaying the screen during the execution of the stress test.
This component will receive its \texttt{settings} and an \texttt{onFinished} callback inside its props. 
The \texttt{settings} object contains the settings configured for this screen. 
The \texttt{onFinished} callback must be called when the screen is finished performing its task, so the pipeline executor can show the next screen. 
This callback must only be called once. 
A very simple screen component might look like this:

\begin{verbatim}
export default function ScreenComponent(props) {
  const {settings, onFinished} = props;
  return (<Button onClick={onFinished}>{settings.text}</Button>);
}
\end{verbatim}

\subsection{Overlay Component}
\label{sec:development-overlays}
In this section, you will learn how to code an overlay's main component. 
This React component is responsible for displaying the overlay during the execution of the stress test. 
It will receive only its settings inside the props. 
A very simple overlay that displays a message might look like this:
\begin{verbatim}
export default function OverlayComponent(props) {
  return (<div>{props.message}</div>);
}
\end{verbatim}

By default, an overlay is configured to not receive any click events, because the current screen must handle them. 
If an overlay component wants to receive click events it can enable them by adding the \texttt{pointer-events: all;} CSS property to an overlay HTML element. 
But be careful as this can make the underlying screen components unclickable.

\subsubsection*{Position of overlays}
Overlays can be positioned on the whole screen. 
For easier positioning of overlays, overlays can define an optional \texttt{position} property in its settings. 
This property accepts the following values: 
\texttt{center}, \texttt{top}, \texttt{bottom}, \texttt{right}, \texttt{left}, \texttt{top-left}, \texttt{top-right}, \texttt{bottom-left}, \texttt{bottom-right}.
To make the position property adjustable a \nameref{sec:component-overlay-position-input} component exists. 
Please refer to the \nameref{sec:settings-component} and \nameref{sec:input-components} section for further information.

\subsection{Settings Component}
\label{sec:settings-component}
In this section you will learn how to create the settings React component for screen and overlay modules.
The settings component for screen and overlay modules are working the same way. 
The settings component will receive the following properties in its props.
\begin{itemize}
  \item \texttt{id}: The unique id of this pipeline element
  \item \texttt{dndType}: The drag and drop type (either \texttt{screen} or \texttt{overlay})
  \item \texttt{type}: The name of this module
  \item \texttt{updateSettings} callback: This function must be called to update a settings value. 
    It takes the id of the pipeline element, the name of the settings value and the new value
  \item all current settings
\end{itemize}

There are many predefined input components available. 
For details about them refer to the \nameref{sec:input-components} section. 
An example settings component with an number input might look like this:
\begin{verbatim}
export default function SettingsComponent(props) {
  const onChange = 
    (name, value) => props.updateSettings(props.id, name, value);
  return (
    <NumberInput 
      name="test" 
      label="..." 
      value={props.test} 
      onChange={onChange} 
    />
  );
}
\end{verbatim}

\subsection{How to register a module}
Each module must be registered in the corresponding registry so it can be used. 
Screen modules must be registered in the \texttt{screen-registry.js} file and overlay modules in the \texttt{overlay-registry.js}. 
Each registry exports a map of modules where the key is the unique module name and the value is the definition of the module. 
This definition has the following properties:
\begin{itemize}
  \item \texttt{component}: The overlay's or screen's main React component
  \item \texttt{settingsComponent}: The overlay's or screen's settings React component
  \item \texttt{initialSettings}: An object containing the initial/default settings
\end{itemize}

Here is an example registry file with one module named \texttt{moduleName}:
\begin{verbatim}
export default {
  moduleName: {
    component: ModuleMainComponent,
    initialSettings: {
      test: "Foo"
    },
    settingsComponent: ModuleSettingsComponent,
  },
};
\end{verbatim}

\subsection{Input Components}
\label{sec:input-components}
Input component can be used to create settings components for the modules.
One input component can be used to change one settings property.
There are some input components already defined and they all have the following properties:
\begin{itemize}
  \item \texttt{name}: the name of the settings property this input component controls
  \item \texttt{value}: the current value of the settings property
  \item \texttt{onChange}: callback that is called when the value changed. 
    The callback will be called with two arguments.
    The first is the name of the property (passed via the \texttt{name} property) and the second is the new value.
  \item \texttt{label}: The description of the settings property controlled by this input component.
\end{itemize}

In the following example for a \nameref{sec:component-number-input}, you can see that the name \texttt{testTotalTime} matches the corresponding setting name read from \texttt{props.testTotalTime}:
\begin{verbatim}
<NumberInput
  name="testTotalTime"
  label={t("settings.totalTime")}
  value={props.testTotalTime}
  onChange={onChange}
/> 
\end{verbatim}

\subsubsection*{Input validation}
To validate the input of the users, validation attributes can be added to the input component. 
If the value of the input is not valid an error message will be displayed and the input component will get surrounded by a red box. 
The following validation attributes are available:
\begin{itemize}
  \item \texttt{required}: The value can not be empty.
  \item \texttt{minLength}: Only for text values: The text length must be greater or equal to the attribute's value.
  \item \texttt{maxLength}: Only for text values: The text length must be lower or equal to the attribute's value.
  \item \texttt{pattern}: Only for text values: The text must match the given regular expression.
  \item \texttt{min}: Only for number values: The number must be greater or equal to the attribute's value.
  \item \texttt{max}: Only for number values: The number must be lower or equal to the attribute's value.
\end{itemize}

The following example shows the usage of the input validation attributes. 
This \nameref{sec:component-number-input} can not be empty and only integers between $-5$ and $5$ (both inclusive) are considered as valid.
Other attributes of the input component are omitted for brevity.
\begin{verbatim}
<NumberInput required min={-5} max={5} />
\end{verbatim}

\subsubsection*{Available input components}
\paragraph{Checkbox}
\label{sec:component-checkbox}
Input component for a boolean value.

\paragraph{MultilineTextInput}
\label{sec:component-multiline-text-input}
Input component for a multiline text value. 
There also exists the \nameref{sec:component-text-input} component which accepts only single line text.

\paragraph{NumberInput}
\label{sec:component-number-input}
Input component for an integer value. 
It also accepts negative integers.

\paragraph{SelectInput}
\label{sec:component-select-input}
Input component to select one value from a list of given values. 
The array of possible values must be passed to the select input via the \texttt{values} property.

\paragraph{TextInput}
\label{sec:component-text-input}
Input component for a single-line text value. 
It does not accept the newline character. 
There also exists the \nameref{sec:component-multiline-text-input} component, which accepts multiline text values.

\paragraph{OverlayPositionInput}
\label{sec:component-overlay-position-input}
Input component to adjust the position property for overlays. 
This component acts as a \nameref{sec:component-select-input} with the list of position values already configured.
